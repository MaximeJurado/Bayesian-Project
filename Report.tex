\documentclass[]{article}
\usepackage{lmodern}
\usepackage{amssymb,amsmath}
\usepackage{ifxetex,ifluatex}
\usepackage{fixltx2e} % provides \textsubscript
\usepackage{fancyhdr}
\pagestyle{fancy}
\ifnum 0\ifxetex 1\fi\ifluatex 1\fi=0 % if pdftex
  \usepackage[T1]{fontenc}
  \usepackage[utf8]{inputenc}
\else % if luatex or xelatex
  \ifxetex
    \usepackage{mathspec}
    \usepackage{xltxtra,xunicode}
  \else
    \usepackage{fontspec}
  \fi
  \defaultfontfeatures{Mapping=tex-text,Scale=MatchLowercase}
  \newcommand{\euro}{€}
\fi
% use upquote if available, for straight quotes in verbatim environments
\IfFileExists{upquote.sty}{\usepackage{upquote}}{}
% use microtype if available
\IfFileExists{microtype.sty}{%
\usepackage{microtype}
\UseMicrotypeSet[protrusion]{basicmath} % disable protrusion for tt fonts
}{}
\usepackage[margin=1in]{geometry}
\usepackage{color}
\usepackage{fancyvrb}
\newcommand{\VerbBar}{|}
\newcommand{\VERB}{\Verb[commandchars=\\\{\}]}
\DefineVerbatimEnvironment{Highlighting}{Verbatim}{commandchars=\\\{\}}
% Add ',fontsize=\small' for more characters per line
\usepackage{framed}
\definecolor{shadecolor}{RGB}{248,248,248}
\newenvironment{Shaded}{\begin{snugshade}}{\end{snugshade}}
\newcommand{\KeywordTok}[1]{\textcolor[rgb]{0.13,0.29,0.53}{\textbf{{#1}}}}
\newcommand{\DataTypeTok}[1]{\textcolor[rgb]{0.13,0.29,0.53}{{#1}}}
\newcommand{\DecValTok}[1]{\textcolor[rgb]{0.00,0.00,0.81}{{#1}}}
\newcommand{\BaseNTok}[1]{\textcolor[rgb]{0.00,0.00,0.81}{{#1}}}
\newcommand{\FloatTok}[1]{\textcolor[rgb]{0.00,0.00,0.81}{{#1}}}
\newcommand{\CharTok}[1]{\textcolor[rgb]{0.31,0.60,0.02}{{#1}}}
\newcommand{\StringTok}[1]{\textcolor[rgb]{0.31,0.60,0.02}{{#1}}}
\newcommand{\CommentTok}[1]{\textcolor[rgb]{0.56,0.35,0.01}{\textit{{#1}}}}
\newcommand{\OtherTok}[1]{\textcolor[rgb]{0.56,0.35,0.01}{{#1}}}
\newcommand{\AlertTok}[1]{\textcolor[rgb]{0.94,0.16,0.16}{{#1}}}
\newcommand{\FunctionTok}[1]{\textcolor[rgb]{0.00,0.00,0.00}{{#1}}}
\newcommand{\RegionMarkerTok}[1]{{#1}}
\newcommand{\ErrorTok}[1]{\textbf{{#1}}}
\newcommand{\NormalTok}[1]{{#1}}
\ifxetex
  \usepackage[setpagesize=false, % page size defined by xetex
              unicode=false, % unicode breaks when used with xetex
              xetex]{hyperref}
\else
  \usepackage[unicode=true]{hyperref}
\fi
\hypersetup{breaklinks=true,
            bookmarks=true,
            colorlinks=true,
            citecolor=blue,
            urlcolor=blue,
            linkcolor=magenta,
            pdfborder={0 0 0}}
\urlstyle{same}  % don't use monospace font for urls
\setlength{\parindent}{0pt}
\setlength{\parskip}{6pt plus 2pt minus 1pt}
\setlength{\emergencystretch}{3em}  % prevent overfull lines
\setcounter{secnumdepth}{0}

%%% Use protect on footnotes to avoid problems with footnotes in titles
\let\rmarkdownfootnote\footnote%
\def\footnote{\protect\rmarkdownfootnote}

%%% Change title format to be more compact
\usepackage{titling}
\setlength{\droptitle}{-2em}
  \title{Salaries in the industry, trade and services in 2011}
  \pretitle{\vspace{\droptitle}\centering\huge}
  \posttitle{\par}
  \author{Maxime Jurado and Mathieu Marauri}
  \preauthor{\centering\large\emph}
  \postauthor{\par}
  \date{}
  \predate{}\postdate{}


\usepackage{float} \usepackage{graphicx}


\begin{document}


\begin{titlepage}

\mbox{}
\begin{center}
\Huge{Salaries in the industry, trade and services in 2011 in France}
\end{center}
\vspace*{1 in}

\vspace*{2cm}\mbox{}

\begin{center}
\includegraphics[width=12cm]{logo_upc.jpg}
\end{center}
\smallskip
\begin{center}
\Large
\begin{tabular}{c}
\sc Facultat de Matem\`atiques i Estad\'istica\\
\sc M\`aster en Estad\'istica i Investigaci\'o Operativa \\

\end{tabular}
\normalsize
\end{center}
\smallskip

\vspace{3cm}

\large
  \begin{flushright}
    \vspace*{1cm}\mbox{}
$
\begin{array}{lcl}
\mbox{Author} &:& \mbox{Maxime Jurado and Mathieu Marauri}\\
\end{array}
$
\end{flushright}

\vspace*{0.5 in}

\begin{center}
Barcelona, Spain
\end{center}
\normalsize
\end{titlepage}

\newpage


{
\hypersetup{linkcolor=black}
\setcounter{tocdepth}{2}
\tableofcontents
}
\newpage

\section{Abstract}\label{abstract}

We analyzed the impact of the gender, socio-professional categories and
time (part-time / full-time job) on the salary. We performed Bayesian
Gaussian Regression and Gaussian hierarchical Bayesian regression (with
random effects) to carry this study. Both of them leaded to same
conclusion. If we order the socio-professional categories from 1 to 5, 1
being considered the ``best'', people from category 1 will earn more
than people in 5. We also see that men earn more than women.

\section{Introduction}\label{introduction}

The structure of the salary in the industry, the services or the trade
is a matter of importance in the society. It can reflects some
inequalities, regarding the gender for instance, or some tendencies. The
aim of this study is to analyse the effect of several indicators such as
the Socio-Professional Category or the gender on the salary.

Performing such an analysis on data coming from France is for us really
interesting because we have several preconceptions about the impact of
the gender for example. Studying those data will allow us to either
confirm or deny these preconceptions.

By performing Bayesian models the effects of the indicators will be
known. Those effects will be quantified. It will provide useful
knowledge on the structure of the salary.

\section{Dataset}\label{dataset}

\label{dataset}

The dataset comes from the \emph{Institut National de la Statistique et
des Etudes Economiques} or INSEE in France. It is the statistical
institute of France. The dataset contains 33 sectors of activity that
are identified by the \emph{id} variable. The classification can be
found in table \ref{tableIndexSectors}.

The variables reported in this study are listed below:

\begin{itemize}
\itemsep1pt\parskip0pt\parsep0pt
\item
  Time is an indicator which takes value 1 for a full-time work and 0
  for a part-time work. \emph{time}
\item
  Sexe is a binary variable with value 1 for male and 0 for female.
  \emph{sexe}
\item
  Socio-Professional Category: it takes values between 1 and 5. Table
  \ref{tableIndexSpc} shows the classification. \emph{spc}
\item
  Salary: it is the response variable. \emph{salary}
\end{itemize}

The salary is the average gross income for an hour. For instance it can
be the average income of female employees working full-time in the
extraction industry.

In the original dataset there were a cell by combination of the
different variables. It was built so that no cell contains less than 5
entries and that no entry would represent more than 80\% of the total.
Some data are missing due to statistical privacy. It is the case for the
id 8 which is the sector of Transportation equipment manufacture.

\newpage

\section{Statistical methods}\label{statistical-methods}

In this study several Bayesian models will be performed, first a
Gaussian Bayesian model then a hierarchical Bayesian model. In both
cases covariates will be add to an initial model and by comparing the
Deviance Information Criterion or DIC. Since a model with more
parameters will always be preferable to a ``simpler'' one, parsimony
will also be used to select the best model. The goal is to have the best
model but also the one that is

\subsection{Gaussian Bayesian
regression}\label{gaussian-bayesian-regression}

In the Gaussian Bayesian model the point is to get the mean effect of
the covariates on the salary. By performing such a model one wants to
know how the covariates impact the salary. Among the different models
that can be performed the best one will be used as the initial model in
the hierarchical Bayesian models selection. The Gaussian Bayesian model
is of this form:

\begin{centering}

  $Y|X ~ N(\beta X,\tau_1)$
  
  $\Pi(\beta) ~ N(\mu,\tau_2)$
  
  $\Pi(\tau_1) ~ Gamma(a,b)$

\end{centering}

The parameter \(\beta\) is a vector and X is the matrix of the
covariates. The statistical model Gaussian with a linear expression of
the covariates and the intercept as a mean and a variance of \(\tau_1\).
\(\beta\) also follows a Gaussian distribution. Finally \(\tau_2\)
follows a Gamma with parameters a and b that are fixed to 0.01.

\subsection{Gaussian hierarchical Bayesian
regression}\label{gaussian-hierarchical-bayesian-regression}

A hierarchical Bayesian model is a model with random effects. Those
random effects allow the model to add variance to the estimates of the
parameters of the covariates. It means that each sector has different
values for the estimates of the covariates. This way each sector has
more precise estimates and it is possible to see the differences between
sectors.

The initial model is the Gaussian Bayesian model that was selected. Then
random effects are added to this model and the best model is selected
based on the same selection process than before.

The Gaussian hierarchical model is designed as follow:

\begin{centering}

  $Y|X ~ N(\beta X + b_{1i} X_r + b_{2i},\tau_1)$
  
  $\Pi(\beta) ~ N(\mu,\tau_2)$
  
  $b_i ~N(0,\tau_{3,i})$
  
  $\Pi(\tau_1) ~ Gamma(a_1,b_1)$
  
  $\Pi(\tau_{3,i}) ~ Gamma(a_2,b_2)$

\end{centering}

As previously X is a matrix of the covariates and \(X_r\) is also a
matrix of the covariates but it may be different from X since random
effects can be added on different covariates. Again as before
\(a_1, b_1, a_2 and b_2\) are all equal to 0.01.

\newpage

\section{Descriptive analysis}\label{descriptive-analysis}

An exploration of the data through a descriptive analysis gives first
insights on the behavior of the variables. The point here is to see
graphically if some variables seem to have an impact on the response
variable, the salary. A basic linear regression and some tests give also
some insights.

\subsection{Influence of the variable
\emph{spc}}\label{influence-of-the-variable-spc}

The first idea is that the salary is supposed to be different based on
the different Socio-Professional Categories. The following graph (Figure
\ref{BoxplotCat}) shows the mean of the salary by SPC.

\begin{figure}[H]

\includegraphics{Report_files/figure-latex/boxplotcat-1} \hfill{}

\caption{Boxplot of the salary for each category. \label{BoxplotCat}}\label{fig:boxplotcat}
\end{figure}

One can clearly see that being in the category 1 (Higher managerial and
professional positions) increases the salary as compared to the other
categories. If one considers the categories to be ordered from 1 to 5
then he would be able to say that \emph{SPC} seems to have a negative
impact on the salary.

In order to see more clearly the way \emph{SPC} influences the salary
Figure \ref{completeplot} shows the evolution of the salary for each SPC
for each sector. The means were calculated on all the values of salary
for the SPC by id.

\begin{figure}[H]

\includegraphics{Report_files/figure-latex/completeplot-1} \hfill{}

\caption{Evolution of the salary for each SPC by id. \label{completeplot}}\label{fig:completeplot}
\end{figure}

As seen previously \emph{SPC} seems to have a negative impact on the
salary for every sectors but sector 6 (Manufacture of food, drink and
tobacco based products), sector 21 (Information and communication) and
sector 30 (Arts and entertainment). This may be due to the definitions
of the SPC. In most of the sectors they are ordered from 1 to 5 but in
some other sectors this order can be different.

One can see that the relation between \emph{spc} and \emph{salary} seems
to be quadratic. Therefore it could be useful to add a quadratic part to
the \emph{spc} variable in the next regression models.

\newpage

\subsection{Influence of the variable
\emph{time}}\label{influence-of-the-variable-time}

The salary is given per hour, hence no differences between full-time and
part-time jobs should be observed. The following figure confirms this.

\begin{figure}[H]

\includegraphics{Report_files/figure-latex/boxplottime-1} \hfill{}

\caption{Boxplot of the salary for each time. \label{boxplottime}}\label{fig:boxplottime}
\end{figure}

The salaries for the full-time workers ant the salaries for the
part-time workers seem to be the same.

The same graph was repeated for each category and the following results
were obtained. Figure \ref{boxplotspctime}.

\begin{figure}[H]

\includegraphics{Report_files/figure-latex/boxplotspctime-1} \hfill{}

\caption{Boxplot of the salary for each SPC and each time. \label{boxplotspctime}}\label{fig:boxplotspctime}
\end{figure}

One could say that for some categories a small difference is observed.
For instance salaries for full-time workers in the category of the
higher managerial and professional positions seem to be higher than the
ones for the part-time workers.

A t-test was performed to see if there is a difference between the
salaries of full-time workers and part-time workers. This test was
performed on the whole dataset and then by category. Table
\ref{tabletesttime} shows the results.

\begin{table}[ht]
\centering
\begin{tabular}{ccc}
  \hline
 & p-value & mean of the differences \\ 
  \hline
 dataset & 0.000 & -0.788 \\ 
 category 1 & 0.000 & -1.982 \\ 
 category 2 & 0.298 & -0.277 \\ 
 category 3 & 0.000 & -1.027 \\ 
 category 4 & 0.316 & -0.188 \\ 
 category 5 & 0.051 & -0.427 \\ 
   \hline
\end{tabular}
\caption{P-values and mean of the differences 
             of the tests. \label{tabletesttime}} 
\end{table}

One can see that the p-values for the tests performed on the whole
dataset, on the category 1 and on the category 3 are below 0.05. In
those cases the mean of the differences can be up to almost 2 euros per
hour. It represents a difference of 6.2\%.

\subsection{Influence of the variable
\emph{sexe}}\label{influence-of-the-variable-sexe}

Inequality in the salary regarding the gender is a matter of discussion
in France. The issue is that salaries are said to be higher for males
than they are for females. Therefore some differences are expected.

The means of the salaries for females and males were plotted for each
Socio-Professional Category and it seems that no differences appear
except maybe for the category 1. Figure \ref{barplotsexespc} shows the
results.

\begin{figure}[H]

\includegraphics{Report_files/figure-latex/barplotsexespc-1} \hfill{}

\caption{Barplot of the mean salary by sexe for each category. \label{barplotsexespc}}\label{fig:barplotsexespc}
\end{figure}

For every categories the mean salary for males seems to be higher than
the mean salary for females. A t-test confirmed this result. A
significant mean of the differences of 2 was observed. It means that
males have in mean a salary per hour higher by 2 euros than the one of
the females.

\newpage

\subsection{A linear regression}\label{a-linear-regression}

A basic linear regression was performed to see the impact of the
covariates on the salary. The model that was used is:
\(Salary = \beta_0 + \beta_1time + \beta_2sexe +\beta_3spc + \beta_4sexe*spc + \beta_4*spc^2\)

The following table (Table \ref{tablereg}) shows the estimates and the
p-values obtained with the linear regression model.

\begin{table}[H]
\centering
\begin{tabular}{ccc}
  \hline
 & estimates & p-value \\ 
  \hline
(Intercept) & 40.218 & 0.000 \\ 
  time & 0.788 & 0.000 \\ 
  sexe & 4.370 & 0.000 \\ 
  spc & -14.144 & 0.000 \\ 
  spc² & 1.728 & 0.000 \\ 
  sexe:spc & -0.798 & 0.000 \\ 
   \hline
\end{tabular}
\caption{Results of the linear regression. \label{tablereg}} 
\end{table}

As for the graphical analysis \emph{spc} has a negative impact on the
salary (we consider here that the categories are ordered from 1 to 5).
The variable \emph{sexe} has a quite important positive effect on the
salary. It means that males are paid around 4 euros per hour more than
females. The fact that the interaction between \emph{sexe} and
\emph{spc} has a negative effect on the salary means that the positive
effect of being a male does not counterbalance the negative impact of
the \emph{spc}. Finally \emph{time} has a small positive impact. It
means that when you work in a full-time job, you earn around 0.8 euros
more than a part-time job.

\newpage

\section{Gaussian Bayesian model}\label{gaussian-bayesian-model}

Before going trough the interpretation of a Bayesian Gaussian
Regression, we have to choose which model is the best. For that we start
with a simple model with three covariates: \emph{time}, \emph{sexe} and
\emph{spc}. Then we tried to add \emph{\(spc^2\)} as explained in the
descriptive part. Finally we try to add some interaction between
covariates. To decide which model is the best, we use the DIC criteria
and the error from the prediction. The results are presented below

\begin{table}[ht]
\centering
\begin{tabular}{ccc}
  \hline
 & DIC & res \\ 
  \hline
$\alpha + \beta_{1} time + \beta_2 sexe + \beta_3 spc$ & 3455.60 & 9686.56 \\ 
$\alpha + \beta_{1} time + \beta_2 sexe + \beta_3 spc + \beta_4 spc^2$ & 2988.00 & 4518.68 \\ 
$\alpha + \beta_{1} time + \beta_2 sexe + \beta_3 spc + \beta_4 spc*sexe$ & 3444.80 & 9492.39 \\ 
 $\alpha + \beta_{1} time + \beta_2 sexe + \beta_3 spc + \beta_4 spc*sexe + \beta_5 spc^2$ & 2962.90 & 4324.22 \\ 
   \hline
\end{tabular}
\caption{Table for the selection of the Bayesian models. 
             \label{tableselectionmodel}} 
\end{table}

As you can see, the best model is the one with \emph{\(spc^2\)} and the
interaction between \emph{spc} and \emph{sexe}. However the sum of
squares is still big. It means that our model does not predict well the
future values. That is why we try to add some random effects in the next
part. Now we have selected the model, we can go to the interpretation.

In table \ref{resultsmodel} we can see the results of the Bayesian
Gaussian Regression that we choose above. The coefficients are very
similar to those obtained in the descriptive part. Thus the
interpretation is the same as in \refname{A linear regression}. We still
have this negative impact of \emph{spc} and the interaction between
\emph{sexe} and \emph{spc}. We also find again the positive impact of
\emph{sexe} that does not counterbalance the effect of \emph{spc}.
Finally the impact of \emph{time} is still positive.

\begin{table}[H]
\centering
\begin{tabular}{cc}
  \hline
 & mean \\ 
  \hline
intercept & 40.0549 \\ 
  time & 0.8032 \\ 
  sexe & 4.4220 \\ 
  spc & -14.0450 \\ 
  spc*sexe & -0.8104 \\ 
  spc² & 1.7135 \\ 
   \hline
\end{tabular}
\caption{Mean for the Bayesian model. \label{resultsmodel}} 
\end{table}

\section{Gaussian hierarchical Bayesian
model}\label{gaussian-hierarchical-bayesian-model}

As for the Bayesian Gaussian Regression, we have to choose here the best
model. We proceed the same way, trying to ass random effects to one
covariate at a time. But at the end, we do not want to over-parameterize
the model, that is why we limit this study to 2 random effects maximum.

\begin{table}[ht]
\centering
\begin{tabular}{ccc}
  \hline
 & DIC2 & res2 \\ 
  \hline
$\alpha + \beta_{1} time + \beta_2 sexe + \beta_3 spc + \beta_4 spc*sexe + \beta_5 spc^2$ & 2962.90 & 4324.22 \\ 
 $\alpha + \beta_{1} time + (\beta_2 + b_{i}) sexe + \beta_3 spc + \beta_4 spc*sexe + \beta_5 spc^2$ & 2881.20 & 2279.24 \\ 
 $\alpha + (\beta_{1} + b_{i}) time + \beta_2 sexe + \beta_3 spc + \beta_4 spc*sexe + \beta_5 spc^2$ & 2964.10 & 4048.85 \\ 
 $\alpha + \beta_{1} time + \beta_2 sexe + (\beta_3 + b_{i}) spc + \beta_4 spc*sexe + \beta_5 spc^2$ & 2970.00 & 4208.86 \\ 
 $\alpha + \beta_{1} time + \beta_2 sexe + \beta_3 spc + \beta_4 spc*sexe + (\beta_5 + b_{i}) spc^2$ & 2986.10 & 4076.92 \\ 
 $\alpha + b_{1i} + \beta_{1} time + \beta_2 sexe + \beta_3 spc + \beta_4 spc*sexe + (\beta_5 + b_{2i}) spc^2$ & 2869.30 & 1901.95 \\ 
 $\alpha + b_{1i} + \beta_{1} time + \beta_2 sexe + \beta_3 spc + \beta_4 spc*sexe + (\beta_5 + b_{2i}) spc^2$ & 2682.40 & 1104.37 \\ 
   \hline
\end{tabular}
\caption{Table for the selection of the hierarchical Bayesian models. 
             \label{tableselectionmodel2}} 
\end{table}

You can see that the model with random effects on the \emph{Intercept}
and \emph{sexe} is the best. Furthermore, the sum of squares is lower
than in the Bayesian Gaussian Regression. This is our final model.

In table \ref{resultsmodel2}, the coefficients of our final model are
presented. It can be seen that they do not differ a lot from the
previous interpretations. The main part here is the random effects on
\emph{Intercept} and \emph{sexe}. It means that we add variability for
each subject. We can observe differences of each sector. But here, the
variance added by the random effects does not change the effect of a
covariate. The sign of its coefficient will not change. The global
effect will be the same.

\begin{table}[H]
\centering
\begin{tabular}{cc}
  \hline
 & mean \\ 
  \hline
intercept & 39.6747 \\ 
  time & 0.6752 \\ 
  sexe & 4.3804 \\ 
  spc & -13.6565 \\ 
  spc*sexe & -0.7962 \\ 
  spc² & 1.6482 \\ 
   \hline
\end{tabular}
\caption{Mean for the hierarchical model. \label{resultsmodel2}} 
\end{table}

\newpage

\section{Conclusion}\label{conclusion}

The aim of this study was to describe the structure of the salary per
hour in France by performing Bayesian models. After a descriptive
analysis that gave us insights on the structure of the salary we
performed several Bayesian regressions. The selection process led us to
select a hierarchical model over a simple Bayesian model. Based on all
of this it appeared that the Socio-Professional Category has a great
impact on the salary; being in the first category -Higher managerial and
professional positions- insures a better salary. The second major effect
that we observed is the one of the gender. Males have higher salary than
women. Those results were observed both in Gaussian Bayesian regressions
and in hierarchical Gaussian Bayesian regressions.

As a further analysis one could try to do a cluster analysis in order to
regroup the sectors that are similar in the way the salary is
structured. Some data could be add to the dataset in order to specify
the structure of the salary even more. For instance information on the
age and on the number of workers by cells.

\vspace{3cm}

\section{Bibliography}\label{bibliography}

\begin{itemize}
\itemsep1pt\parskip0pt\parsep0pt
\item
  \href{http://www.insee.fr/fr/themes/detail.asp?reg_id=0\&ref_id=ir-irsocdads2011\&page=irweb/irsocdads2011/dd/irsocdads2011_t10.htm}{INSEE
  T101 Salaire brut horaire, par âge et catégorie socioprofessionnelle
  simplifiée}
\item
  \href{http://homepage.stat.uiowa.edu/~gwoodwor/BBIText/AppendixBWinbugs.pdf}{Introduction
  fo WinBUGS}
\item
  \href{https://www.uclouvain.be/cps/ucl/doc/stat/documents/WinBUGSTraining_smcs.pdf}{WinBUGS
  Training}
\end{itemize}

\newpage

\section{Appendix}\label{appendix}

\subsection{Classification tables}\label{classification-tables}

Here are presented the classification tables for the Socio-Professional
Categories (Table \ref{tableIndexSpc}) and for the Sectors (Table
\ref{tableIndexSectors}).

\begin{table}[H]
\centering
\begin{tabular}{cc}
  \hline
 & Category \\ 
  \hline
1 & Higher managerial and professional positions  \\ 
  2 & Intermediate occupations \\ 
  3 & Employee \\ 
  4 & Skilled worker \\ 
  5 & Unskilled worker \\ 
   \hline
\end{tabular}
\caption{Classification table of the Socio-Professional Categories. \label{tableIndexSpc}} 
\end{table}

\begin{table}[H]
\centering
\begin{tabular}{cc}
  \hline
 & Sector \\ 
  \hline
1 & Agriculture, silviculture and fishing \\ 
  2 & Manufacturing industry, extraction industry \\ 
  3 & Extraction industry, energy, water, trash management and remediation \\ 
  4 & Extraction industry \\ 
  5 & Water production and distribution, remediation, trash management \\ 
  6 & Manufacture of food, drink and tobacco based products \\ 
  7 & Electric, electronic and computer components manufacture \\ 
  8 & Transportation equipment manufacture \\ 
  9 & Other industrial products manufacture \\ 
  10 & Textile, clothing industry, leather and shoe industry \\ 
  11 & Wood working, paper and printing industry \\ 
  12 & Manufacture of rubber and plastic materials and other non metallic products \\ 
  13 & Metalworking industry and production of metallic products except machine and equipment \\ 
  14 & Other manufacture industry, repair and installation of machines and equipment \\ 
  15 & Construction \\ 
  16 & Wholesale trade and retailing, transport, accommodation and restoration \\ 
  17 & Trade, automobile and motorcycle repairs \\ 
  18 & Transportation and stocking \\ 
  19 & Accommodation and food service industry \\ 
  20 & Diverse services \\ 
  21 & Information and communication \\ 
  22 & Publishing, audiovisual media and airing \\ 
  23 & Computer activities and information services \\ 
  24 & Financial and insurance activities \\ 
  25 & Scientific and technical activities, administrative and support services \\ 
  26 & Legal, accounting, management, architecture, control and technical analysis activities \\ 
  27 & Other specialized activities, scientific and technical \\ 
  28 & Administrative services and support activities \\ 
  29 & Other service activities \\ 
  30 & Arts and entertainment \\ 
  31 & Other service activities  \\ 
  32 & Public administration, teaching, health and social action \\ 
  33 & Medico-social accommodation, social, and social action without accommodation \\ 
   \hline
\end{tabular}
\caption{Classification table of the Sectors. \label{tableIndexSectors}} 
\end{table}

Return to section \nameref{dataset}.

\subsection{Code}\label{code}

\subsubsection{WinBUGS code}\label{winbugs-code}

\begin{Shaded}
\begin{Highlighting}[]
\NormalTok{model\{}
\NormalTok{for(i in }\DecValTok{1}\NormalTok{:n)\{}
\NormalTok{salary[i]~}\KeywordTok{dnorm}\NormalTok{(mu[i],tau)}
\NormalTok{mu[i] <-}\StringTok{ }\NormalTok{alpha +}\StringTok{ }\NormalTok{beta1*time[i] +}\StringTok{ }\NormalTok{beta2*sexe[i] +}\StringTok{ }\NormalTok{beta3*spc[i] +}\StringTok{ }\NormalTok{beta4*spc[i]*sexe[i] +}\StringTok{ }
\NormalTok{beta5*spc2[i]}
\NormalTok{\}}
\NormalTok{tau~}\KeywordTok{dgamma}\NormalTok{(}\FloatTok{0.01}\NormalTok{,}\FloatTok{0.01}\NormalTok{)}
\NormalTok{alpha~}\KeywordTok{dnorm}\NormalTok{(}\DecValTok{0}\NormalTok{,}\FloatTok{0.01}\NormalTok{)}
\NormalTok{beta1~}\KeywordTok{dnorm}\NormalTok{(}\DecValTok{0}\NormalTok{,}\FloatTok{0.01}\NormalTok{)}
\NormalTok{beta2~}\KeywordTok{dnorm}\NormalTok{(}\DecValTok{0}\NormalTok{,}\FloatTok{0.01}\NormalTok{)}
\NormalTok{beta3~}\KeywordTok{dnorm}\NormalTok{(}\DecValTok{0}\NormalTok{,}\FloatTok{0.01}\NormalTok{)}
\NormalTok{beta4~}\KeywordTok{dnorm}\NormalTok{(}\DecValTok{0}\NormalTok{,}\FloatTok{0.01}\NormalTok{)}
\NormalTok{beta5~}\KeywordTok{dnorm}\NormalTok{(}\DecValTok{0}\NormalTok{,}\FloatTok{0.01}\NormalTok{)}
\NormalTok{\}}

\NormalTok{###}

\NormalTok{model\{}
\NormalTok{for(i in }\DecValTok{1}\NormalTok{:n)\{}
\NormalTok{salary[i]~}\KeywordTok{dnorm}\NormalTok{(mu[i],tau)}
\NormalTok{mu[i] <-}\StringTok{ }\NormalTok{alpha +}\StringTok{ }\NormalTok{beta1*time[i] +}\StringTok{ }\NormalTok{(beta2+b[i])*sexe[i] +}\StringTok{ }\NormalTok{beta3*spc[i] +}\StringTok{ }\NormalTok{beta4*spc[i]*sexe[i] +}
\NormalTok{beta5*spc2[i] +}\StringTok{ }\NormalTok{b1[i]}
\NormalTok{b[i]~}\KeywordTok{dnorm}\NormalTok{(}\DecValTok{0}\NormalTok{,tau2)}
\NormalTok{b1[i]~}\KeywordTok{dnorm}\NormalTok{(}\DecValTok{0}\NormalTok{,tau3)}
\NormalTok{\}}
\NormalTok{tau~}\KeywordTok{dgamma}\NormalTok{(}\FloatTok{0.01}\NormalTok{,}\FloatTok{0.01}\NormalTok{)}
\NormalTok{alpha~}\KeywordTok{dnorm}\NormalTok{(}\DecValTok{0}\NormalTok{,}\FloatTok{0.01}\NormalTok{)}
\NormalTok{beta1~}\KeywordTok{dnorm}\NormalTok{(}\DecValTok{0}\NormalTok{,}\FloatTok{0.01}\NormalTok{)}
\NormalTok{beta2~}\KeywordTok{dnorm}\NormalTok{(}\DecValTok{0}\NormalTok{,}\FloatTok{0.01}\NormalTok{)}
\NormalTok{beta3~}\KeywordTok{dnorm}\NormalTok{(}\DecValTok{0}\NormalTok{,}\FloatTok{0.01}\NormalTok{)}
\NormalTok{beta4~}\KeywordTok{dnorm}\NormalTok{(}\DecValTok{0}\NormalTok{,}\FloatTok{0.01}\NormalTok{)}
\NormalTok{beta5~}\KeywordTok{dnorm}\NormalTok{(}\DecValTok{0}\NormalTok{,}\FloatTok{0.01}\NormalTok{)}
\NormalTok{tau2~}\KeywordTok{dgamma}\NormalTok{(}\FloatTok{0.01}\NormalTok{, }\FloatTok{0.01}\NormalTok{)}
\NormalTok{tau3~}\KeywordTok{dgamma}\NormalTok{(}\FloatTok{0.01}\NormalTok{, }\FloatTok{0.01}\NormalTok{)}
\NormalTok{\}}
\end{Highlighting}
\end{Shaded}

\subsubsection{R code}\label{r-code}

\begin{Shaded}
\begin{Highlighting}[]
\CommentTok{# Packages ----------------------------------------------------------------}

\KeywordTok{library}\NormalTok{(}\StringTok{"ggplot2"}\NormalTok{)}
\KeywordTok{library}\NormalTok{(}\StringTok{"xtable"}\NormalTok{)}


\CommentTok{# Data --------------------------------------------------------------------}

\NormalTok{data <-}\StringTok{ }\KeywordTok{read.csv2}\NormalTok{(}\StringTok{"Data/data.csv"}\NormalTok{,}\DataTypeTok{header=}\OtherTok{TRUE}\NormalTok{)}
\NormalTok{data <-}\StringTok{ }\KeywordTok{subset}\NormalTok{(data,time!=}\StringTok{"TP"}\NormalTok{)}

\NormalTok{data1 <-}\StringTok{ }\NormalTok{data[}\KeywordTok{rep}\NormalTok{(}\DecValTok{1}\NormalTok{:}\KeywordTok{nrow}\NormalTok{(data),}\DataTypeTok{each=}\DecValTok{5}\NormalTok{),]}
\NormalTok{data2 <-}\StringTok{ }\KeywordTok{data.frame}\NormalTok{(data$X1,data$X2,data$X3,data$X4,data$X5)}
\NormalTok{data3 <-}\StringTok{ }\KeywordTok{data.frame}\NormalTok{(data1$id,data1$time,data1$sexe)}

\NormalTok{vec <-}\StringTok{ }\OtherTok{NULL}
\NormalTok{for(i in }\DecValTok{1}\NormalTok{:}\KeywordTok{nrow}\NormalTok{(data2))\{}
  \NormalTok{vec1 <-}\StringTok{ }\NormalTok{data2[i,]}
  \NormalTok{vec <-}\StringTok{ }\KeywordTok{c}\NormalTok{(vec,}\KeywordTok{t}\NormalTok{(vec1))}
\NormalTok{\}}

\NormalTok{vec <-}\StringTok{ }\KeywordTok{as.data.frame}\NormalTok{(vec)}
\NormalTok{data4 <-}\StringTok{ }\KeywordTok{data.frame}\NormalTok{(data3,vec)}
\NormalTok{cat <-}\StringTok{ }\KeywordTok{c}\NormalTok{(}\KeywordTok{rep}\NormalTok{(}\KeywordTok{c}\NormalTok{(}\DecValTok{1}\NormalTok{,}\DecValTok{2}\NormalTok{,}\DecValTok{3}\NormalTok{,}\DecValTok{4}\NormalTok{,}\DecValTok{5}\NormalTok{),}\DecValTok{132}\NormalTok{))}

\NormalTok{data <-}\StringTok{ }\KeywordTok{data.frame}\NormalTok{(data4,cat)}
\KeywordTok{colnames}\NormalTok{(data) <-}\StringTok{ }\KeywordTok{c}\NormalTok{(}\StringTok{"id"}\NormalTok{,}\StringTok{"time"}\NormalTok{,}\StringTok{"sexe"}\NormalTok{,}\StringTok{"salary"}\NormalTok{,}\StringTok{"spc"}\NormalTok{)}
\KeywordTok{attach}\NormalTok{(data)}
\NormalTok{salary<-}\KeywordTok{as.numeric}\NormalTok{(salary)}
\NormalTok{sexe <-}\StringTok{ }\KeywordTok{as.factor}\NormalTok{(sexe)}
\NormalTok{time <-}\StringTok{ }\KeywordTok{as.factor}\NormalTok{(time)}
\NormalTok{spc <-}\StringTok{ }\KeywordTok{as.factor}\NormalTok{(spc)}


\CommentTok{# Descriptive analysis ----------------------------------------------------}

\NormalTok{theme <-}\StringTok{ }\KeywordTok{theme}\NormalTok{(}\DataTypeTok{plot.title =} \KeywordTok{element_text}\NormalTok{(}\DataTypeTok{size=}\DecValTok{16}\NormalTok{, }\DataTypeTok{face=}\StringTok{"bold"}\NormalTok{), }\DataTypeTok{axis.title.x =} \KeywordTok{element_text}\NormalTok{(}
              \DataTypeTok{size=}\DecValTok{15}\NormalTok{), }\DataTypeTok{axis.title.y =} \KeywordTok{element_text}\NormalTok{(}\DataTypeTok{size=}\DecValTok{15}\NormalTok{), }\DataTypeTok{axis.text.x =} \KeywordTok{element_text}\NormalTok{(}
              \DataTypeTok{size=}\DecValTok{15}\NormalTok{, }\DataTypeTok{face=}\StringTok{"bold"}\NormalTok{),}\DataTypeTok{axis.text.y =} \KeywordTok{element_text}\NormalTok{(}\DataTypeTok{size=}\DecValTok{15}\NormalTok{, }\DataTypeTok{face=}\StringTok{"bold"}\NormalTok{))}

\CommentTok{# SPC}
\KeywordTok{ggplot}\NormalTok{(data, }\KeywordTok{aes}\NormalTok{(}\KeywordTok{factor}\NormalTok{(spc), salary)) +}\StringTok{ }\KeywordTok{geom_boxplot}\NormalTok{(}\KeywordTok{aes}\NormalTok{(}\DataTypeTok{fill =} \KeywordTok{factor}\NormalTok{(spc))) +}\StringTok{ }
\StringTok{  }\KeywordTok{xlab}\NormalTok{(}\StringTok{"Socio-Professional Category"}\NormalTok{) +}\StringTok{ }\KeywordTok{ylab}\NormalTok{(}\StringTok{"Salary"}\NormalTok{) +}\StringTok{ }\KeywordTok{guides}\NormalTok{(}\DataTypeTok{fill=}\KeywordTok{guide_legend}\NormalTok{(}\DataTypeTok{title=}\OtherTok{NULL}\NormalTok{)) +}\StringTok{ }
\StringTok{  }\KeywordTok{ggtitle}\NormalTok{(}\StringTok{"Boxplot of the salary for each category"}\NormalTok{) +}\StringTok{ }\NormalTok{theme}

\NormalTok{datamean <-}\StringTok{ }\KeywordTok{aggregate}\NormalTok{(}\KeywordTok{data.frame}\NormalTok{(}\DataTypeTok{salaryMean =} \NormalTok{data$salary), }\DataTypeTok{by =} \KeywordTok{list}\NormalTok{(}\DataTypeTok{id =} \NormalTok{data$id, }\DataTypeTok{spc =} \NormalTok{data$spc), }
                      \NormalTok{mean)}

\KeywordTok{ggplot}\NormalTok{(datamean, }\KeywordTok{aes}\NormalTok{(spc, salaryMean)) +}\StringTok{ }\KeywordTok{facet_wrap}\NormalTok{(}\DataTypeTok{facets=}\NormalTok{~}\StringTok{ }\NormalTok{id, }\DataTypeTok{ncol=}\DecValTok{10}\NormalTok{) +}\StringTok{ }
\StringTok{  }\KeywordTok{geom_line}\NormalTok{(}\DataTypeTok{colour=}\StringTok{"blue"}\NormalTok{) +}\StringTok{ }\KeywordTok{geom_point}\NormalTok{(}\DataTypeTok{colour=}\StringTok{"red"}\NormalTok{) +}\StringTok{ }\NormalTok{theme}

\CommentTok{# Time}
\KeywordTok{ggplot}\NormalTok{(data, }\KeywordTok{aes}\NormalTok{(}\KeywordTok{factor}\NormalTok{(time), salary)) +}\StringTok{ }\KeywordTok{geom_boxplot}\NormalTok{(}\KeywordTok{aes}\NormalTok{(}\DataTypeTok{fill =} \KeywordTok{factor}\NormalTok{(time))) +}
\StringTok{  }\KeywordTok{labs}\NormalTok{(}\DataTypeTok{x=}\StringTok{"Time"}\NormalTok{, }\DataTypeTok{y=}\StringTok{"Salary"}\NormalTok{, }\DataTypeTok{title=}\StringTok{"Boxplot of the salary for each time"}\NormalTok{) +}\StringTok{ }\NormalTok{theme +}
\StringTok{  }\KeywordTok{scale_fill_manual}\NormalTok{(}\DataTypeTok{name=}\StringTok{"Time"}\NormalTok{, }\DataTypeTok{values=}\KeywordTok{c}\NormalTok{(}\StringTok{"orange"}\NormalTok{, }\StringTok{"mediumpurple"}\NormalTok{), }\DataTypeTok{labels=}\KeywordTok{c}\NormalTok{(}\StringTok{"0"}\NormalTok{=}\StringTok{"Part-time"}\NormalTok{, }
  \StringTok{"1"}\NormalTok{=}\StringTok{"Full-time"}\NormalTok{))}
  
\KeywordTok{ggplot}\NormalTok{(data, }\KeywordTok{aes}\NormalTok{(}\KeywordTok{factor}\NormalTok{(spc), salary)) +}\StringTok{ }\KeywordTok{geom_boxplot}\NormalTok{(}\KeywordTok{aes}\NormalTok{(}\DataTypeTok{fill =} \KeywordTok{factor}\NormalTok{(time))) +}
\StringTok{  }\KeywordTok{labs}\NormalTok{(}\DataTypeTok{x=}\StringTok{"SPC"}\NormalTok{, }\DataTypeTok{y=}\StringTok{"Salary"}\NormalTok{, }\DataTypeTok{title=}\StringTok{"Boxplot of the salary for each SPC and each time"}\NormalTok{) +}\StringTok{ }
\StringTok{  }\NormalTok{theme +}\StringTok{ }\KeywordTok{scale_fill_manual}\NormalTok{(}\DataTypeTok{name=}\StringTok{"Time"}\NormalTok{, }\DataTypeTok{values=}\KeywordTok{c}\NormalTok{(}\StringTok{"orange"}\NormalTok{, }\StringTok{"mediumpurple"}\NormalTok{), }\DataTypeTok{labels=}\KeywordTok{c}\NormalTok{(}\StringTok{"0"}\NormalTok{=}\StringTok{"Part-time"}\NormalTok{, }
  \StringTok{"1"}\NormalTok{=}\StringTok{"Full-time"}\NormalTok{))}

\NormalTok{## Test on the differences between time1 and time0.}
\NormalTok{data.sort <-}\StringTok{ }\NormalTok{data[}\KeywordTok{with}\NormalTok{(data, }\KeywordTok{order}\NormalTok{(id, spc)), ]}
\KeywordTok{row.names}\NormalTok{(data.sort) <-}\StringTok{ }\KeywordTok{c}\NormalTok{(}\DecValTok{1}\NormalTok{:}\KeywordTok{nrow}\NormalTok{(data.sort))}
\NormalTok{data.sort.time0 <-}\StringTok{ }\OtherTok{NULL}
\NormalTok{data.sort.time1 <-}\StringTok{ }\OtherTok{NULL}
\NormalTok{for(i in }\DecValTok{1}\NormalTok{:}\KeywordTok{nrow}\NormalTok{(data.sort))\{}
  \NormalTok{newline <-}\StringTok{ }\NormalTok{data.sort[i,]}
  \NormalTok{if(i%%}\DecValTok{2}\NormalTok{==}\DecValTok{0}\NormalTok{)\{}
    \NormalTok{data.sort.time0 <-}\StringTok{ }\KeywordTok{rbind}\NormalTok{(data.sort.time0, newline)}
  \NormalTok{\}}
  \NormalTok{else\{}
    \NormalTok{data.sort.time1 <-}\StringTok{ }\KeywordTok{rbind}\NormalTok{(data.sort.time1, newline)}
  \NormalTok{\}}
\NormalTok{\}}
\NormalTok{data.time.paired <-}\StringTok{ }\KeywordTok{as.data.frame}\NormalTok{(}\KeywordTok{cbind}\NormalTok{(data.sort.time0$spc,data.sort.time0[,}\DecValTok{4}\NormalTok{],data.sort.time1[,}\DecValTok{4}\NormalTok{]))}
\KeywordTok{colnames}\NormalTok{(data.time.paired) <-}\StringTok{ }\KeywordTok{c}\NormalTok{(}\StringTok{"spc"}\NormalTok{, }\StringTok{"salary0"}\NormalTok{, }\StringTok{"salary1"}\NormalTok{)}

\NormalTok{test.time.all <-}\StringTok{ }\KeywordTok{t.test}\NormalTok{(data.time.paired[,}\DecValTok{2}\NormalTok{], data.time.paired[,}\DecValTok{3}\NormalTok{], }\DataTypeTok{paired=}\OtherTok{TRUE}\NormalTok{)}

\NormalTok{data.time.paired.spc1 <-}\StringTok{ }\KeywordTok{subset}\NormalTok{(data.time.paired, spc==}\DecValTok{1}\NormalTok{)}
\NormalTok{data.time.paired.spc2 <-}\StringTok{ }\KeywordTok{subset}\NormalTok{(data.time.paired, spc==}\DecValTok{2}\NormalTok{)}
\NormalTok{data.time.paired.spc3 <-}\StringTok{ }\KeywordTok{subset}\NormalTok{(data.time.paired, spc==}\DecValTok{3}\NormalTok{)}
\NormalTok{data.time.paired.spc4 <-}\StringTok{ }\KeywordTok{subset}\NormalTok{(data.time.paired, spc==}\DecValTok{4}\NormalTok{)}
\NormalTok{data.time.paired.spc5 <-}\StringTok{ }\KeywordTok{subset}\NormalTok{(data.time.paired, spc==}\DecValTok{5}\NormalTok{)}

\NormalTok{test.time.spc1 <-}\StringTok{ }\KeywordTok{t.test}\NormalTok{(data.time.paired.spc1[,}\DecValTok{2}\NormalTok{], data.time.paired.spc1[,}\DecValTok{3}\NormalTok{], }\DataTypeTok{paired=}\OtherTok{TRUE}\NormalTok{)}
\NormalTok{test.time.spc2 <-}\StringTok{ }\KeywordTok{t.test}\NormalTok{(data.time.paired.spc2[,}\DecValTok{2}\NormalTok{], data.time.paired.spc2[,}\DecValTok{3}\NormalTok{], }\DataTypeTok{paired=}\OtherTok{TRUE}\NormalTok{)}
\NormalTok{test.time.spc3 <-}\StringTok{ }\KeywordTok{t.test}\NormalTok{(data.time.paired.spc3[,}\DecValTok{2}\NormalTok{], data.time.paired.spc3[,}\DecValTok{3}\NormalTok{], }\DataTypeTok{paired=}\OtherTok{TRUE}\NormalTok{)}
\NormalTok{test.time.spc4 <-}\StringTok{ }\KeywordTok{t.test}\NormalTok{(data.time.paired.spc4[,}\DecValTok{2}\NormalTok{], data.time.paired.spc4[,}\DecValTok{3}\NormalTok{], }\DataTypeTok{paired=}\OtherTok{TRUE}\NormalTok{)}
\NormalTok{test.time.spc5 <-}\StringTok{ }\KeywordTok{t.test}\NormalTok{(data.time.paired.spc5[,}\DecValTok{2}\NormalTok{], data.time.paired.spc5[,}\DecValTok{3}\NormalTok{], }\DataTypeTok{paired=}\OtherTok{TRUE}\NormalTok{)}

\NormalTok{table.test.time <-}\StringTok{ }\KeywordTok{data.frame}\NormalTok{(}\KeywordTok{c}\NormalTok{(}\StringTok{"dataset"}\NormalTok{,}\StringTok{"category 1"}\NormalTok{,}\StringTok{"category 2"}\NormalTok{,}\StringTok{"category 3"}\NormalTok{,}\StringTok{"category 4"}\NormalTok{,}
                   \StringTok{"category 5"}\NormalTok{),}\KeywordTok{c}\NormalTok{(test.time.all$p.value,test.time.spc1$p.value,}
                   \NormalTok{test.time.spc2$p.value,test.time.spc3$p.value,test.time.spc4$p.value,}
                   \NormalTok{test.time.spc5$p.value), }\KeywordTok{c}\NormalTok{(test.time.all$estimate,test.time.spc1$estimate,}
                   \NormalTok{test.time.spc2$estimate,test.time.spc3$estimate,test.time.spc4$estimate,}
                   \NormalTok{test.time.spc5$estimate))}
\KeywordTok{colnames}\NormalTok{(table.test.time) <-}\StringTok{ }\KeywordTok{c}\NormalTok{(}\StringTok{""}\NormalTok{,}\StringTok{"p-value"}\NormalTok{,}\StringTok{"mean of the differences"}\NormalTok{)}
\KeywordTok{print}\NormalTok{(}\KeywordTok{xtable}\NormalTok{(table.test.time, }\DataTypeTok{align=}\KeywordTok{c}\NormalTok{(}\StringTok{"c"}\NormalTok{,}\StringTok{"c"}\NormalTok{,}\StringTok{"c"}\NormalTok{,}\StringTok{"c"}\NormalTok{), }\DataTypeTok{caption=}\StringTok{"P-values and mean of the differences }
\StringTok{             of the tests. }\CharTok{\textbackslash{}\textbackslash{}}\StringTok{label\{tabletesttime\}"}\NormalTok{, }\DataTypeTok{digits=}\DecValTok{3}\NormalTok{))}

\NormalTok{salary.mean.spc <-}\StringTok{ }\KeywordTok{aggregate}\NormalTok{(}\KeywordTok{data.frame}\NormalTok{(}\DataTypeTok{salaryMean=}\NormalTok{data$salary),}\DataTypeTok{by=}\KeywordTok{list}\NormalTok{(}\DataTypeTok{time=}\NormalTok{data$time,}
                             \DataTypeTok{spc=}\NormalTok{data$spc),mean,}\DataTypeTok{na.rm=}\OtherTok{TRUE}\NormalTok{)}
\NormalTok{(table.test.time[}\DecValTok{2}\NormalTok{,}\DecValTok{3}\NormalTok{]/salary.mean.spc[}\DecValTok{2}\NormalTok{,}\DecValTok{3}\NormalTok{])*}\DecValTok{100}

\CommentTok{# Sexe}
\NormalTok{data.m <-}\StringTok{ }\KeywordTok{aggregate}\NormalTok{(}\KeywordTok{data.frame}\NormalTok{(}\DataTypeTok{salary.m =} \NormalTok{data$salary), }\DataTypeTok{by =} \KeywordTok{list}\NormalTok{(}\DataTypeTok{sexe =} \NormalTok{data$sexe, }
                    \DataTypeTok{spc =} \NormalTok{data$spc,}\DataTypeTok{time =} \NormalTok{data$time), mean, }\DataTypeTok{na.rm=}\OtherTok{TRUE}\NormalTok{)}

\KeywordTok{ggplot}\NormalTok{(}\KeywordTok{subset}\NormalTok{(data.m,time==}\DecValTok{0}\NormalTok{),}\KeywordTok{aes}\NormalTok{(}\DataTypeTok{x=}\KeywordTok{factor}\NormalTok{(sexe),}\DataTypeTok{y=}\NormalTok{salary.m,}\DataTypeTok{fill=}\KeywordTok{factor}\NormalTok{(sexe))) +}\StringTok{ }
\StringTok{  }\KeywordTok{geom_bar}\NormalTok{(}\DataTypeTok{stat =} \StringTok{"identity"}\NormalTok{) +}\StringTok{ }\KeywordTok{facet_wrap}\NormalTok{(~}\StringTok{ }\NormalTok{spc) +}\StringTok{ }
\StringTok{  }\KeywordTok{labs}\NormalTok{(}\DataTypeTok{title=}\StringTok{"Mean salary by gender and spc at time=0 (Part-Time)"}\NormalTok{, }\DataTypeTok{x=}\StringTok{"Sexe"}\NormalTok{, }\DataTypeTok{y=}\StringTok{"Mean salary"}\NormalTok{) +}
\StringTok{  }\KeywordTok{scale_fill_discrete}\NormalTok{(}\DataTypeTok{name=}\StringTok{"Gender"}\NormalTok{,  }\DataTypeTok{labels=}\KeywordTok{c}\NormalTok{(}\StringTok{"Female"}\NormalTok{, }\StringTok{"Male"}\NormalTok{),}\DataTypeTok{guide =} \KeywordTok{guide_legend}\NormalTok{(}\DataTypeTok{reverse=}\OtherTok{TRUE}\NormalTok{)) +}\StringTok{ }
\StringTok{  }\KeywordTok{theme}\NormalTok{(}\DataTypeTok{plot.title =} \KeywordTok{element_text}\NormalTok{(}\DataTypeTok{size=}\DecValTok{16}\NormalTok{, }\DataTypeTok{face=}\StringTok{"bold"}\NormalTok{), }\DataTypeTok{legend.title =} \KeywordTok{element_text}\NormalTok{(}\DataTypeTok{colour=}\StringTok{"black"}\NormalTok{,}
  \DataTypeTok{size=}\DecValTok{18}\NormalTok{, }\DataTypeTok{face=}\StringTok{"bold"}\NormalTok{),}\DataTypeTok{legend.position=}\KeywordTok{c}\NormalTok{(.}\DecValTok{85}\NormalTok{,.}\DecValTok{15}\NormalTok{),}\DataTypeTok{legend.background =} \KeywordTok{element_rect}\NormalTok{(}\DataTypeTok{size=}\DecValTok{25}\NormalTok{),}
  \DataTypeTok{legend.text =} \KeywordTok{element_text}\NormalTok{(}\DataTypeTok{size =} \DecValTok{16}\NormalTok{))}
 
\KeywordTok{ggplot}\NormalTok{(}\KeywordTok{subset}\NormalTok{(data.m,time==}\DecValTok{0}\NormalTok{),}\KeywordTok{aes}\NormalTok{(}\DataTypeTok{x=}\KeywordTok{factor}\NormalTok{(sexe),}\DataTypeTok{y=}\NormalTok{salary.m,}\DataTypeTok{fill=}\KeywordTok{factor}\NormalTok{(sexe))) +}\StringTok{ }
\StringTok{  }\KeywordTok{geom_bar}\NormalTok{(}\DataTypeTok{stat =} \StringTok{"identity"}\NormalTok{) +}\StringTok{ }\KeywordTok{facet_wrap}\NormalTok{(~}\StringTok{ }\NormalTok{spc) +}\StringTok{ }
\StringTok{  }\KeywordTok{labs}\NormalTok{(}\DataTypeTok{title=}\StringTok{"Mean salary by gender and spc at time=1 (Full Time)"}\NormalTok{, }\DataTypeTok{x=}\StringTok{"Sexe"}\NormalTok{, }\DataTypeTok{y=}\StringTok{"Mean salary"}\NormalTok{) +}
\StringTok{  }\KeywordTok{scale_fill_discrete}\NormalTok{(}\DataTypeTok{name=}\StringTok{"Gender"}\NormalTok{,  }\DataTypeTok{labels=}\KeywordTok{c}\NormalTok{(}\StringTok{"Female"}\NormalTok{, }\StringTok{"Male"}\NormalTok{),}\DataTypeTok{guide =} \KeywordTok{guide_legend}\NormalTok{(}\DataTypeTok{reverse=}\OtherTok{TRUE}\NormalTok{)) +}\StringTok{ }
\StringTok{  }\KeywordTok{theme}\NormalTok{(}\DataTypeTok{plot.title =} \KeywordTok{element_text}\NormalTok{(}\DataTypeTok{size=}\DecValTok{16}\NormalTok{, }\DataTypeTok{face=}\StringTok{"bold"}\NormalTok{),}\DataTypeTok{legend.title =} \KeywordTok{element_text}\NormalTok{(}\DataTypeTok{colour=}\StringTok{"black"}\NormalTok{,}
  \DataTypeTok{size=}\DecValTok{18}\NormalTok{, }\DataTypeTok{face=}\StringTok{"bold"}\NormalTok{),}\DataTypeTok{legend.position=}\KeywordTok{c}\NormalTok{(.}\DecValTok{85}\NormalTok{,.}\DecValTok{15}\NormalTok{),}\DataTypeTok{legend.background =} \KeywordTok{element_rect}\NormalTok{(}\DataTypeTok{size=}\DecValTok{25}\NormalTok{),}
  \DataTypeTok{legend.text =} \KeywordTok{element_text}\NormalTok{(}\DataTypeTok{size =} \DecValTok{16}\NormalTok{))}

\KeywordTok{ggplot}\NormalTok{(data.m,}\KeywordTok{aes}\NormalTok{(}\DataTypeTok{x=}\KeywordTok{factor}\NormalTok{(sexe),}\DataTypeTok{y=}\NormalTok{salary.m,}\DataTypeTok{fill=}\KeywordTok{factor}\NormalTok{(sexe))) +}\StringTok{ }\KeywordTok{geom_bar}\NormalTok{(}\DataTypeTok{stat =} \StringTok{"identity"}\NormalTok{) +}\StringTok{ }
\StringTok{  }\KeywordTok{facet_wrap}\NormalTok{(~}\StringTok{ }\NormalTok{spc) +}\StringTok{ }\KeywordTok{labs}\NormalTok{(}\DataTypeTok{title=}\StringTok{"Mean salary by gender and spc"}\NormalTok{, }\DataTypeTok{x=}\StringTok{"Sexe"}\NormalTok{, }\DataTypeTok{y=}\StringTok{"Mean salary"}\NormalTok{) +}\StringTok{ }
\StringTok{  }\KeywordTok{scale_fill_discrete}\NormalTok{(}\DataTypeTok{name=}\StringTok{"Gender"}\NormalTok{,}\DataTypeTok{labels=}\KeywordTok{c}\NormalTok{(}\StringTok{"Female"}\NormalTok{,}\StringTok{"Male"}\NormalTok{),}\DataTypeTok{guide =} \KeywordTok{guide_legend}\NormalTok{(}\DataTypeTok{reverse=}\OtherTok{TRUE}\NormalTok{)) +}
\StringTok{  }\KeywordTok{theme}\NormalTok{(}\DataTypeTok{plot.title =} \KeywordTok{element_text}\NormalTok{(}\DataTypeTok{size=}\DecValTok{16}\NormalTok{, }\DataTypeTok{face=}\StringTok{"bold"}\NormalTok{),}\DataTypeTok{legend.title =} \KeywordTok{element_text}\NormalTok{(}\DataTypeTok{colour=}\StringTok{"black"}\NormalTok{,}
  \DataTypeTok{size=}\DecValTok{18}\NormalTok{, }\DataTypeTok{face=}\StringTok{"bold"}\NormalTok{),}\DataTypeTok{legend.position=}\KeywordTok{c}\NormalTok{(.}\DecValTok{85}\NormalTok{,.}\DecValTok{15}\NormalTok{),}\DataTypeTok{legend.background =} \KeywordTok{element_rect}\NormalTok{(}\DataTypeTok{size=}\DecValTok{25}\NormalTok{),}
  \DataTypeTok{legend.text =} \KeywordTok{element_text}\NormalTok{(}\DataTypeTok{size =} \DecValTok{16}\NormalTok{))}

\NormalTok{## Test on the differences between time1 and time0.}
\NormalTok{data.sort2 <-}\StringTok{ }\NormalTok{data[}\KeywordTok{with}\NormalTok{(data, }\KeywordTok{order}\NormalTok{(id, spc, time)), ]}
\KeywordTok{row.names}\NormalTok{(data.sort2) <-}\StringTok{ }\KeywordTok{c}\NormalTok{(}\DecValTok{1}\NormalTok{:}\KeywordTok{nrow}\NormalTok{(data.sort2))}
\NormalTok{data.sort.sexe0 <-}\StringTok{ }\OtherTok{NULL}
\NormalTok{data.sort.sexe1 <-}\StringTok{ }\OtherTok{NULL}
\NormalTok{for(i in }\DecValTok{1}\NormalTok{:}\KeywordTok{nrow}\NormalTok{(data.sort2))\{}
  \NormalTok{newline <-}\StringTok{ }\NormalTok{data.sort2[i,]}
  \NormalTok{if(i%%}\DecValTok{2}\NormalTok{==}\DecValTok{0}\NormalTok{)\{}
    \NormalTok{data.sort.sexe0 <-}\StringTok{ }\KeywordTok{rbind}\NormalTok{(data.sort.sexe0, newline)}
  \NormalTok{\}}
  \NormalTok{else\{}
    \NormalTok{data.sort.sexe1 <-}\StringTok{ }\KeywordTok{rbind}\NormalTok{(data.sort.sexe1, newline)}
  \NormalTok{\}}
\NormalTok{\}}
\NormalTok{data.sexe.paired <-}\StringTok{ }\KeywordTok{as.data.frame}\NormalTok{(}\KeywordTok{cbind}\NormalTok{(data.sort.sexe0$spc,data.sort.sexe0[,}\DecValTok{4}\NormalTok{],data.sort.sexe1[,}\DecValTok{4}\NormalTok{]))}
\KeywordTok{colnames}\NormalTok{(data.sexe.paired) <-}\StringTok{ }\KeywordTok{c}\NormalTok{(}\StringTok{"spc"}\NormalTok{, }\StringTok{"salary0"}\NormalTok{, }\StringTok{"salary1"}\NormalTok{)}

\NormalTok{test.sexe.all <-}\StringTok{ }\KeywordTok{t.test}\NormalTok{(data.sexe.paired[,}\DecValTok{2}\NormalTok{], data.sexe.paired[,}\DecValTok{3}\NormalTok{], }\DataTypeTok{paired=}\OtherTok{TRUE}\NormalTok{)}

\CommentTok{# Linear regression}
\NormalTok{data.spc2 <-}\StringTok{ }\KeywordTok{cbind}\NormalTok{(data,data$spc^}\DecValTok{2}\NormalTok{)}
\KeywordTok{colnames}\NormalTok{(data.spc2) <-}\StringTok{ }\KeywordTok{c}\NormalTok{(}\StringTok{"id"}\NormalTok{,}\StringTok{"time"}\NormalTok{,}\StringTok{"sexe"}\NormalTok{,}\StringTok{"salary"}\NormalTok{,}\StringTok{"spc"}\NormalTok{,}\StringTok{"spc2"}\NormalTok{)}

\NormalTok{reg <-}\StringTok{ }\KeywordTok{lm}\NormalTok{(salary~time+sexe*spc, }\DataTypeTok{data=}\NormalTok{data)}
\KeywordTok{summary}\NormalTok{(reg)}

\NormalTok{reg.spc2 <-}\StringTok{ }\KeywordTok{lm}\NormalTok{(salary~time+sexe*spc+spc2, }\DataTypeTok{data=}\NormalTok{data.spc2)}
\KeywordTok{summary}\NormalTok{(reg.spc2)}

\KeywordTok{anova}\NormalTok{(reg,reg.spc2)}

\NormalTok{table.reg <-}\StringTok{ }\KeywordTok{data.frame}\NormalTok{(}\KeywordTok{coefficients}\NormalTok{(reg.spc2),}\KeywordTok{summary}\NormalTok{(reg.spc2)[[}\DecValTok{4}\NormalTok{]][,}\DecValTok{4}\NormalTok{])}
\KeywordTok{colnames}\NormalTok{(table.reg) <-}\StringTok{ }\KeywordTok{c}\NormalTok{(}\StringTok{"estimates"}\NormalTok{,}\StringTok{"p-value"}\NormalTok{)}


\CommentTok{# Missings ----------------------------------------------------------------}

\NormalTok{dataMissing <-}\StringTok{ }\NormalTok{data[}\KeywordTok{which}\NormalTok{(}\KeywordTok{is.na}\NormalTok{(data$salary)),]}

\KeywordTok{length}\NormalTok{(dataMissing)}

\NormalTok{id1 <-}\StringTok{ }\NormalTok{data[}\KeywordTok{which}\NormalTok{(data$id==}\DecValTok{1}\NormalTok{),]}
\CommentTok{# Missing in the data}
\NormalTok{id8 <-}\StringTok{ }\NormalTok{data[}\KeywordTok{which}\NormalTok{(data$id==}\DecValTok{8}\NormalTok{),]}
\CommentTok{# Secret data}
\NormalTok{id22 <-}\StringTok{ }\NormalTok{data[}\KeywordTok{which}\NormalTok{(data$id==}\DecValTok{22}\NormalTok{),]}
\CommentTok{# Missing in the data}


\CommentTok{# Bayesian model ----------------------------------------------------------}

\KeywordTok{library}\NormalTok{(R2WinBUGS)}

\NormalTok{path.bug <-}\StringTok{ "C:/Users/Mathieu/Documents/Cours/2A/Erasmus/Cours/Bayesian analysis/Bayesian-Project/modelBug/"}
\NormalTok{path.WBS <-}\StringTok{ "C:/Users/Mathieu/Documents/Logiciels/WinBuggs/WinBUGS14/"}

\NormalTok{Iter <-}\StringTok{ }\DecValTok{1000}
\NormalTok{Burn <-}\StringTok{ }\DecValTok{500}
\NormalTok{Chain <-}\StringTok{ }\DecValTok{2}
\NormalTok{Thin <-}\StringTok{ }\DecValTok{1}
\NormalTok{n <-}\StringTok{ }\KeywordTok{nrow}\NormalTok{(data)}

\NormalTok{datalist <-}\StringTok{ }\KeywordTok{list}\NormalTok{(}\DataTypeTok{salary=}\NormalTok{data$salary, }\DataTypeTok{time=}\NormalTok{data$time, }\DataTypeTok{sexe=}\NormalTok{data$sexe, }\DataTypeTok{spc=}\NormalTok{data$spc, }\DataTypeTok{n=}\NormalTok{n)}
\NormalTok{datalist2 <-}\StringTok{ }\KeywordTok{list}\NormalTok{(}\DataTypeTok{salary=}\NormalTok{data$salary,}\DataTypeTok{time=}\NormalTok{data$time, }\DataTypeTok{sexe=}\NormalTok{data$sexe, }\DataTypeTok{spc=}\NormalTok{data$spc, }
                  \DataTypeTok{spc2=}\NormalTok{(data$spc)^}\DecValTok{2}\NormalTok{, }\DataTypeTok{n=}\NormalTok{n)}

\CommentTok{# sexe,time and spc}
\NormalTok{parameters10 <-}\StringTok{ }\KeywordTok{c}\NormalTok{(}\StringTok{"alpha"}\NormalTok{,}\StringTok{"beta1"}\NormalTok{,}\StringTok{"beta2"}\NormalTok{,}\StringTok{"beta3"}\NormalTok{,}\StringTok{"tau"}\NormalTok{,}\StringTok{"mu"}\NormalTok{)}
\NormalTok{inits10 <-}\StringTok{ }\KeywordTok{list}\NormalTok{(}\KeywordTok{list}\NormalTok{(}\DataTypeTok{tau=}\DecValTok{1}\NormalTok{),}\KeywordTok{list}\NormalTok{(}\DataTypeTok{tau=}\DecValTok{5}\NormalTok{))}
\NormalTok{model10 <-}\StringTok{ }\KeywordTok{bugs}\NormalTok{(datalist,}\DataTypeTok{inits=}\NormalTok{inits10,}\DataTypeTok{parameters.to.save=}\NormalTok{parameters10,}
               \DataTypeTok{model=}\KeywordTok{paste}\NormalTok{(path.bug,}\StringTok{"modelWin10.bug"}\NormalTok{,}\DataTypeTok{sep=}\StringTok{""}\NormalTok{),}\DataTypeTok{bugs.directory=}\NormalTok{path.WBS,}
               \DataTypeTok{n.iter=}\NormalTok{(Iter*Thin+Burn),}\DataTypeTok{n.burnin=}\NormalTok{Burn,}\DataTypeTok{n.thin=}\NormalTok{Thin,}\DataTypeTok{n.chains=}\NormalTok{Chain, }\DataTypeTok{DIC=}\NormalTok{F,}\DataTypeTok{debug=}\NormalTok{T)}
\KeywordTok{print}\NormalTok{(model10, }\DataTypeTok{digits=}\DecValTok{4}\NormalTok{)}
\CommentTok{# DIC=3455.6}

\CommentTok{# Sexe, time, spc and spc²}
\NormalTok{parameters11 <-}\StringTok{ }\KeywordTok{c}\NormalTok{(}\StringTok{"alpha"}\NormalTok{, }\StringTok{"beta1"}\NormalTok{, }\StringTok{"beta2"}\NormalTok{, }\StringTok{"beta3"}\NormalTok{,}\StringTok{"beta4"}\NormalTok{, }\StringTok{"tau"}\NormalTok{, }\StringTok{"mu"}\NormalTok{)}
\NormalTok{inits11 <-}\StringTok{ }\KeywordTok{list}\NormalTok{(}\KeywordTok{list}\NormalTok{(}\DataTypeTok{tau=}\DecValTok{1}\NormalTok{), }\KeywordTok{list}\NormalTok{(}\DataTypeTok{tau=}\DecValTok{5}\NormalTok{))}
\NormalTok{model11 <-}\StringTok{ }\KeywordTok{bugs}\NormalTok{(datalist2, }\DataTypeTok{inits=}\NormalTok{inits11, }\DataTypeTok{parameters.to.save=}\NormalTok{parameters11,}
                \DataTypeTok{model=}\KeywordTok{paste}\NormalTok{(path.bug,}\StringTok{"modelWin11.bug"}\NormalTok{,}\DataTypeTok{sep=}\StringTok{""}\NormalTok{),}
                \DataTypeTok{bugs.directory=}\NormalTok{path.WBS,               }
                \DataTypeTok{n.iter=}\NormalTok{(Iter*Thin+Burn),}\DataTypeTok{n.burnin=}\NormalTok{Burn, }\DataTypeTok{n.thin=}\NormalTok{Thin, }\DataTypeTok{n.chains=}\NormalTok{Chain, }\DataTypeTok{DIC=}\NormalTok{T, }\DataTypeTok{debug=}\NormalTok{T)}
\KeywordTok{print}\NormalTok{(model11, }\DataTypeTok{digits=}\DecValTok{4}\NormalTok{)}
\CommentTok{# DIC=2987.5}

\CommentTok{# sexe, time, spc and spc*sexe}
\NormalTok{parameters12 <-}\StringTok{ }\KeywordTok{c}\NormalTok{(}\StringTok{"alpha"}\NormalTok{, }\StringTok{"beta1"}\NormalTok{, }\StringTok{"beta2"}\NormalTok{, }\StringTok{"beta3"}\NormalTok{,}\StringTok{"beta4"}\NormalTok{, }\StringTok{"tau"}\NormalTok{, }\StringTok{"mu"}\NormalTok{)}
\NormalTok{inits12 <-}\StringTok{ }\KeywordTok{list}\NormalTok{(}\KeywordTok{list}\NormalTok{(}\DataTypeTok{tau=}\DecValTok{1}\NormalTok{), }\KeywordTok{list}\NormalTok{(}\DataTypeTok{tau=}\DecValTok{5}\NormalTok{))}
\NormalTok{model12 <-}\StringTok{ }\KeywordTok{bugs}\NormalTok{(datalist, }\DataTypeTok{inits=}\NormalTok{inits12, }\DataTypeTok{parameters.to.save=}\NormalTok{parameters12,}
                \DataTypeTok{model=}\KeywordTok{paste}\NormalTok{(path.bug,}\StringTok{"modelWin12.bug"}\NormalTok{,}\DataTypeTok{sep=}\StringTok{""}\NormalTok{),}
                \DataTypeTok{bugs.directory=}\NormalTok{path.WBS,               }
                \DataTypeTok{n.iter=}\NormalTok{(Iter*Thin+Burn),}\DataTypeTok{n.burnin=}\NormalTok{Burn, }\DataTypeTok{n.thin=}\NormalTok{Thin, }\DataTypeTok{n.chains=}\NormalTok{Chain, }\DataTypeTok{DIC=}\NormalTok{T, }\DataTypeTok{debug=}\NormalTok{T)}
\KeywordTok{print}\NormalTok{(model12, }\DataTypeTok{digits=}\DecValTok{4}\NormalTok{)}
\CommentTok{# DIC=3444.8}

\CommentTok{# sexe, time, spc, spc*sexe and spc²}
\NormalTok{parameters13 <-}\StringTok{ }\KeywordTok{c}\NormalTok{(}\StringTok{"alpha"}\NormalTok{, }\StringTok{"beta1"}\NormalTok{, }\StringTok{"beta2"}\NormalTok{, }\StringTok{"beta3"}\NormalTok{,}\StringTok{"beta4"}\NormalTok{,}\StringTok{"beta5"}\NormalTok{, }\StringTok{"tau"}\NormalTok{, }\StringTok{"mu"}\NormalTok{)}
\NormalTok{inits13 <-}\StringTok{ }\KeywordTok{list}\NormalTok{(}\KeywordTok{list}\NormalTok{(}\DataTypeTok{tau=}\DecValTok{1}\NormalTok{), }\KeywordTok{list}\NormalTok{(}\DataTypeTok{tau=}\DecValTok{5}\NormalTok{))}
\NormalTok{model13 <-}\StringTok{ }\KeywordTok{bugs}\NormalTok{(datalist2, }\DataTypeTok{inits=}\NormalTok{inits13, }\DataTypeTok{parameters.to.save=}\NormalTok{parameters13,}
             \DataTypeTok{model=}\KeywordTok{paste}\NormalTok{(path.bug,}\StringTok{"modelWin13.bug"}\NormalTok{,}\DataTypeTok{sep=}\StringTok{""}\NormalTok{),}
             \DataTypeTok{bugs.directory=}\NormalTok{path.WBS,               }
             \DataTypeTok{n.iter=}\NormalTok{(Iter*Thin+Burn),}\DataTypeTok{n.burnin=}\NormalTok{Burn, }\DataTypeTok{n.thin=}\NormalTok{Thin, }\DataTypeTok{n.chains=}\NormalTok{Chain, }\DataTypeTok{DIC=}\NormalTok{T, }\DataTypeTok{debug=}\NormalTok{T)}
\KeywordTok{print}\NormalTok{(model13, }\DataTypeTok{digits=}\DecValTok{4}\NormalTok{) }
\CommentTok{# DIC=2962.9}

\NormalTok{results <-}\StringTok{ }\KeywordTok{t}\NormalTok{(}\KeywordTok{as.data.frame}\NormalTok{(}\KeywordTok{c}\NormalTok{(model13$mean[}\DecValTok{1}\NormalTok{],model13$mean[}\DecValTok{2}\NormalTok{],model13$mean[}\DecValTok{3}\NormalTok{],model13$mean[}\DecValTok{4}\NormalTok{],}
             \NormalTok{model13$mean[}\DecValTok{5}\NormalTok{],model13$mean[}\DecValTok{6}\NormalTok{])))}
\KeywordTok{colnames}\NormalTok{(results) <-}\StringTok{ }\KeywordTok{c}\NormalTok{(}\StringTok{"mean"}\NormalTok{)}
\KeywordTok{rownames}\NormalTok{(results) <-}\StringTok{ }\KeywordTok{c}\NormalTok{(}\StringTok{"intercept"}\NormalTok{, }\StringTok{"time"}\NormalTok{, }\StringTok{"sexe"}\NormalTok{, }\StringTok{"spc"}\NormalTok{, }\StringTok{"spc*sexe"}\NormalTok{, }\StringTok{"spc²"}\NormalTok{)}

\NormalTok{DIC <-}\StringTok{ }\KeywordTok{c}\NormalTok{(}\FloatTok{3455.6}\NormalTok{,}\FloatTok{2988.0}\NormalTok{,}\FloatTok{3444.8}\NormalTok{,}\FloatTok{2962.9}\NormalTok{)}

\NormalTok{pred10 <-}\StringTok{ }\NormalTok{model10$mean$mu}
\NormalTok{res10 <-}\StringTok{ }\NormalTok{pred10 -}\StringTok{ }\NormalTok{salary}
\NormalTok{s10 <-}\StringTok{ }\KeywordTok{sum}\NormalTok{(res10^}\DecValTok{2}\NormalTok{,}\DataTypeTok{na.rm=}\OtherTok{TRUE}\NormalTok{)}

\NormalTok{pred11 <-}\StringTok{ }\NormalTok{model11$mean$mu}
\NormalTok{res11 <-}\StringTok{ }\NormalTok{pred11 -}\StringTok{ }\NormalTok{salary}
\NormalTok{s11 <-}\StringTok{ }\KeywordTok{sum}\NormalTok{(res11^}\DecValTok{2}\NormalTok{,}\DataTypeTok{na.rm=}\OtherTok{TRUE}\NormalTok{)}

\NormalTok{pred12 <-}\StringTok{ }\NormalTok{model12$mean$mu}
\NormalTok{res12 <-}\StringTok{ }\NormalTok{pred12 -}\StringTok{ }\NormalTok{salary}
\NormalTok{s12 <-}\StringTok{ }\KeywordTok{sum}\NormalTok{(res12^}\DecValTok{2}\NormalTok{,}\DataTypeTok{na.rm=}\OtherTok{TRUE}\NormalTok{)}

\NormalTok{pred13 <-}\StringTok{ }\NormalTok{model13$mean$mu}
\NormalTok{res13 <-}\StringTok{ }\NormalTok{pred13 -}\StringTok{ }\NormalTok{salary}
\NormalTok{s13 <-}\StringTok{ }\KeywordTok{sum}\NormalTok{(res13^}\DecValTok{2}\NormalTok{,}\DataTypeTok{na.rm=}\OtherTok{TRUE}\NormalTok{)}

\NormalTok{res<-}\KeywordTok{c}\NormalTok{(s10,s11,s12,s13)}

\NormalTok{tab1 <-}\StringTok{ }\KeywordTok{data.frame}\NormalTok{(DIC,res)}

\KeywordTok{print}\NormalTok{(}\KeywordTok{xtable}\NormalTok{(tab1,}\DataTypeTok{align=}\KeywordTok{c}\NormalTok{(}\StringTok{"c"}\NormalTok{,}\StringTok{"c"}\NormalTok{,}\StringTok{"c"}\NormalTok{),}\DataTypeTok{caption=}\StringTok{"Table for the selection of the Bayesian models. }
\StringTok{             }\CharTok{\textbackslash{}\textbackslash{}}\StringTok{label\{tableselectionmodel\}"}\NormalTok{))}


\CommentTok{# Hierarchical Bayesian model ---------------------------------------------}

\CommentTok{# Random effects on time}
\NormalTok{parameters21 <-}\StringTok{ }\KeywordTok{c}\NormalTok{(}\StringTok{"alpha"}\NormalTok{, }\StringTok{"beta1"}\NormalTok{, }\StringTok{"beta2"}\NormalTok{, }\StringTok{"beta3"}\NormalTok{,}\StringTok{"beta4"}\NormalTok{,}\StringTok{"beta5"}\NormalTok{, }\StringTok{"tau"}\NormalTok{,}\StringTok{"tau2"}\NormalTok{, }\StringTok{"mu"}\NormalTok{)}
\NormalTok{inits21 <-}\StringTok{ }\KeywordTok{list}\NormalTok{(}\KeywordTok{list}\NormalTok{(}\DataTypeTok{tau=}\DecValTok{1}\NormalTok{,}\DataTypeTok{tau2=}\DecValTok{1}\NormalTok{), }\KeywordTok{list}\NormalTok{(}\DataTypeTok{tau=}\DecValTok{2}\NormalTok{,}\DataTypeTok{tau2=}\DecValTok{2}\NormalTok{))}
\NormalTok{model21 <-}\StringTok{ }\KeywordTok{bugs}\NormalTok{(datalist2, }\DataTypeTok{inits=}\NormalTok{inits21, }\DataTypeTok{parameters.to.save=}\NormalTok{parameters21,}
               \DataTypeTok{model=}\KeywordTok{paste}\NormalTok{(path.bug,}\StringTok{"modelWin21.bug"}\NormalTok{,}\DataTypeTok{sep=}\StringTok{""}\NormalTok{),}
               \DataTypeTok{bugs.directory=}\NormalTok{path.WBS,               }
               \DataTypeTok{n.iter=}\NormalTok{(Iter*Thin+Burn),}\DataTypeTok{n.burnin=}\NormalTok{Burn, }\DataTypeTok{n.thin=}\NormalTok{Thin, }\DataTypeTok{n.chains=}\NormalTok{Chain, }\DataTypeTok{DIC=}\NormalTok{T, }\DataTypeTok{debug=}\NormalTok{T)}
\KeywordTok{print}\NormalTok{(model21, }\DataTypeTok{digits=}\DecValTok{4}\NormalTok{) }
\CommentTok{# DIC=2964.1}

\CommentTok{# Random effects on sexe}
\NormalTok{parameters22 <-}\StringTok{ }\KeywordTok{c}\NormalTok{(}\StringTok{"alpha"}\NormalTok{, }\StringTok{"beta1"}\NormalTok{, }\StringTok{"beta2"}\NormalTok{, }\StringTok{"beta3"}\NormalTok{,}\StringTok{"beta4"}\NormalTok{,}\StringTok{"beta5"}\NormalTok{, }\StringTok{"tau"}\NormalTok{,}\StringTok{"tau2"}\NormalTok{, }\StringTok{"mu"}\NormalTok{)}
\NormalTok{inits22 <-}\StringTok{ }\KeywordTok{list}\NormalTok{(}\KeywordTok{list}\NormalTok{(}\DataTypeTok{tau=}\DecValTok{1}\NormalTok{,}\DataTypeTok{tau2=}\DecValTok{1}\NormalTok{), }\KeywordTok{list}\NormalTok{(}\DataTypeTok{tau=}\DecValTok{2}\NormalTok{,}\DataTypeTok{tau2=}\DecValTok{2}\NormalTok{))}
\NormalTok{model22 <-}\StringTok{ }\KeywordTok{bugs}\NormalTok{(datalist2, }\DataTypeTok{inits=}\NormalTok{inits22, }\DataTypeTok{parameters.to.save=}\NormalTok{parameters22,}
               \DataTypeTok{model=}\KeywordTok{paste}\NormalTok{(path.bug,}\StringTok{"modelWin22.bug"}\NormalTok{,}\DataTypeTok{sep=}\StringTok{""}\NormalTok{),}
               \DataTypeTok{bugs.directory=}\NormalTok{path.WBS,               }
               \DataTypeTok{n.iter=}\NormalTok{(Iter*Thin+Burn),}\DataTypeTok{n.burnin=}\NormalTok{Burn, }\DataTypeTok{n.thin=}\NormalTok{Thin, }\DataTypeTok{n.chains=}\NormalTok{Chain, }\DataTypeTok{DIC=}\NormalTok{T, }\DataTypeTok{debug=}\NormalTok{T)}
\KeywordTok{print}\NormalTok{(model22, }\DataTypeTok{digits=}\DecValTok{4}\NormalTok{) }
\CommentTok{# DIC=2881.2}

\CommentTok{# Random effects on spc}
\NormalTok{parameters23 <-}\StringTok{ }\KeywordTok{c}\NormalTok{(}\StringTok{"alpha"}\NormalTok{, }\StringTok{"beta1"}\NormalTok{, }\StringTok{"beta2"}\NormalTok{, }\StringTok{"beta3"}\NormalTok{,}\StringTok{"beta4"}\NormalTok{,}\StringTok{"beta5"}\NormalTok{, }\StringTok{"tau"}\NormalTok{,}\StringTok{"tau2"}\NormalTok{, }\StringTok{"mu"}\NormalTok{)}
\NormalTok{inits23 <-}\StringTok{ }\KeywordTok{list}\NormalTok{(}\KeywordTok{list}\NormalTok{(}\DataTypeTok{tau=}\DecValTok{1}\NormalTok{,}\DataTypeTok{tau2=}\DecValTok{1}\NormalTok{), }\KeywordTok{list}\NormalTok{(}\DataTypeTok{tau=}\DecValTok{2}\NormalTok{,}\DataTypeTok{tau2=}\DecValTok{2}\NormalTok{))}
\NormalTok{model23 <-}\StringTok{ }\KeywordTok{bugs}\NormalTok{(datalist2, }\DataTypeTok{inits=}\NormalTok{inits23, }\DataTypeTok{parameters.to.save=}\NormalTok{parameters23,}
               \DataTypeTok{model=}\KeywordTok{paste}\NormalTok{(path.bug,}\StringTok{"modelWin23.bug"}\NormalTok{,}\DataTypeTok{sep=}\StringTok{""}\NormalTok{),}
               \DataTypeTok{bugs.directory=}\NormalTok{path.WBS,               }
               \DataTypeTok{n.iter=}\NormalTok{(Iter*Thin+Burn),}\DataTypeTok{n.burnin=}\NormalTok{Burn, }\DataTypeTok{n.thin=}\NormalTok{Thin, }\DataTypeTok{n.chains=}\NormalTok{Chain, }\DataTypeTok{DIC=}\NormalTok{T, }\DataTypeTok{debug=}\NormalTok{T)}
\KeywordTok{print}\NormalTok{(model23, }\DataTypeTok{digits=}\DecValTok{4}\NormalTok{)}
\CommentTok{# DIC=2970.0}

\CommentTok{# Random effects on spc²}
\NormalTok{parameters24 <-}\StringTok{ }\KeywordTok{c}\NormalTok{(}\StringTok{"alpha"}\NormalTok{, }\StringTok{"beta1"}\NormalTok{, }\StringTok{"beta2"}\NormalTok{, }\StringTok{"beta3"}\NormalTok{,}\StringTok{"beta4"}\NormalTok{,}\StringTok{"beta5"}\NormalTok{, }\StringTok{"tau"}\NormalTok{,}\StringTok{"tau2"}\NormalTok{, }\StringTok{"mu"}\NormalTok{)}
\NormalTok{inits24 <-}\StringTok{ }\KeywordTok{list}\NormalTok{(}\KeywordTok{list}\NormalTok{(}\DataTypeTok{tau=}\DecValTok{1}\NormalTok{,}\DataTypeTok{tau2=}\DecValTok{1}\NormalTok{), }\KeywordTok{list}\NormalTok{(}\DataTypeTok{tau=}\DecValTok{2}\NormalTok{,}\DataTypeTok{tau2=}\DecValTok{2}\NormalTok{))}
\NormalTok{model24 <-}\StringTok{ }\KeywordTok{bugs}\NormalTok{(datalist2, }\DataTypeTok{inits=}\NormalTok{inits24, }\DataTypeTok{parameters.to.save=}\NormalTok{parameters24,}
               \DataTypeTok{model=}\KeywordTok{paste}\NormalTok{(path.bug,}\StringTok{"modelWin24.bug"}\NormalTok{,}\DataTypeTok{sep=}\StringTok{""}\NormalTok{),}
               \DataTypeTok{bugs.directory=}\NormalTok{path.WBS,               }
               \DataTypeTok{n.iter=}\NormalTok{(Iter*Thin+Burn),}\DataTypeTok{n.burnin=}\NormalTok{Burn, }\DataTypeTok{n.thin=}\NormalTok{Thin, }\DataTypeTok{n.chains=}\NormalTok{Chain, }\DataTypeTok{DIC=}\NormalTok{T, }\DataTypeTok{debug=}\NormalTok{T)}
\KeywordTok{print}\NormalTok{(model24, }\DataTypeTok{digits=}\DecValTok{4}\NormalTok{)}
\CommentTok{# DIC=2986.1}

\CommentTok{# Random effects on spc² and the intercept}
\NormalTok{parameters241 <-}\StringTok{ }\KeywordTok{c}\NormalTok{(}\StringTok{"alpha"}\NormalTok{, }\StringTok{"beta1"}\NormalTok{, }\StringTok{"beta2"}\NormalTok{, }\StringTok{"beta3"}\NormalTok{,}\StringTok{"beta4"}\NormalTok{,}\StringTok{"beta5"}\NormalTok{, }\StringTok{"tau"}\NormalTok{,}\StringTok{"tau2"}\NormalTok{, }\StringTok{"tau3"}\NormalTok{, }\StringTok{"mu"}\NormalTok{)}
\NormalTok{inits241 <-}\StringTok{ }\KeywordTok{list}\NormalTok{(}\KeywordTok{list}\NormalTok{(}\DataTypeTok{tau=}\DecValTok{1}\NormalTok{,}\DataTypeTok{tau2=}\DecValTok{1}\NormalTok{, }\DataTypeTok{tau3=}\DecValTok{1}\NormalTok{), }\KeywordTok{list}\NormalTok{(}\DataTypeTok{tau=}\DecValTok{2}\NormalTok{,}\DataTypeTok{tau2=}\DecValTok{2}\NormalTok{, }\DataTypeTok{tau3=}\DecValTok{2}\NormalTok{))}
\NormalTok{model241 <-}\StringTok{ }\KeywordTok{bugs}\NormalTok{(datalist2, }\DataTypeTok{inits=}\NormalTok{inits241, }\DataTypeTok{parameters.to.save=}\NormalTok{parameters241,}
                 \DataTypeTok{model=}\KeywordTok{paste}\NormalTok{(path.bug,}\StringTok{"modelWin241.bug"}\NormalTok{,}\DataTypeTok{sep=}\StringTok{""}\NormalTok{),}
                 \DataTypeTok{bugs.directory=}\NormalTok{path.WBS,               }
                 \DataTypeTok{n.iter=}\NormalTok{(Iter*Thin+Burn),}\DataTypeTok{n.burnin=}\NormalTok{Burn, }\DataTypeTok{n.thin=}\NormalTok{Thin, }\DataTypeTok{n.chains=}\NormalTok{Chain, }\DataTypeTok{DIC=}\NormalTok{T, }\DataTypeTok{debug=}\NormalTok{T)}
\KeywordTok{print}\NormalTok{(model241, }\DataTypeTok{digits=}\DecValTok{4}\NormalTok{) }
\CommentTok{# DIC=2869.3}

\CommentTok{# Random effects sexe and intercept}
\NormalTok{parameters221 <-}\StringTok{ }\KeywordTok{c}\NormalTok{(}\StringTok{"alpha"}\NormalTok{, }\StringTok{"beta1"}\NormalTok{, }\StringTok{"beta2"}\NormalTok{, }\StringTok{"beta3"}\NormalTok{,}\StringTok{"beta4"}\NormalTok{,}\StringTok{"beta5"}\NormalTok{, }\StringTok{"tau"}\NormalTok{,}\StringTok{"tau2"}\NormalTok{, }\StringTok{"tau3"}\NormalTok{, }\StringTok{"mu"}\NormalTok{)}
\NormalTok{inits221 <-}\StringTok{ }\KeywordTok{list}\NormalTok{(}\KeywordTok{list}\NormalTok{(}\DataTypeTok{tau=}\DecValTok{1}\NormalTok{,}\DataTypeTok{tau2=}\DecValTok{1}\NormalTok{, }\DataTypeTok{tau3=}\DecValTok{1}\NormalTok{), }\KeywordTok{list}\NormalTok{(}\DataTypeTok{tau=}\DecValTok{2}\NormalTok{,}\DataTypeTok{tau2=}\DecValTok{2}\NormalTok{, }\DataTypeTok{tau3=}\DecValTok{2}\NormalTok{))}
\NormalTok{model221 <-}\StringTok{ }\KeywordTok{bugs}\NormalTok{(datalist2, }\DataTypeTok{inits=}\NormalTok{inits221, }\DataTypeTok{parameters.to.save=}\NormalTok{parameters221,}
                \DataTypeTok{model=}\KeywordTok{paste}\NormalTok{(path.bug,}\StringTok{"modelWin221.bug"}\NormalTok{,}\DataTypeTok{sep=}\StringTok{""}\NormalTok{),}
                \DataTypeTok{bugs.directory=}\NormalTok{path.WBS,               }
                \DataTypeTok{n.iter=}\NormalTok{(Iter*Thin+Burn),}\DataTypeTok{n.burnin=}\NormalTok{Burn, }\DataTypeTok{n.thin=}\NormalTok{Thin, }\DataTypeTok{n.chains=}\NormalTok{Chain, }\DataTypeTok{DIC=}\NormalTok{T, }\DataTypeTok{debug=}\NormalTok{T)}
\KeywordTok{print}\NormalTok{(model221, }\DataTypeTok{digits=}\DecValTok{4}\NormalTok{) }
\CommentTok{# DIC=2682.4}

\NormalTok{results2 <-}\StringTok{ }\KeywordTok{t}\NormalTok{(}\KeywordTok{as.data.frame}\NormalTok{(}\KeywordTok{c}\NormalTok{(model221$mean[}\DecValTok{1}\NormalTok{],model221$mean[}\DecValTok{2}\NormalTok{],model221$mean[}\DecValTok{3}\NormalTok{],model221$mean[}\DecValTok{4}\NormalTok{],}
                             \NormalTok{model221$mean[}\DecValTok{5}\NormalTok{],model221$mean[}\DecValTok{6}\NormalTok{])))}
\KeywordTok{colnames}\NormalTok{(results2) <-}\StringTok{ }\KeywordTok{c}\NormalTok{(}\StringTok{"mean"}\NormalTok{)}
\KeywordTok{rownames}\NormalTok{(results2) <-}\StringTok{ }\KeywordTok{c}\NormalTok{(}\StringTok{"intercept"}\NormalTok{, }\StringTok{"time"}\NormalTok{, }\StringTok{"sexe"}\NormalTok{, }\StringTok{"spc"}\NormalTok{, }\StringTok{"spc*sexe"}\NormalTok{, }\StringTok{"spc²"}\NormalTok{)}

\NormalTok{DIC2 <-}\StringTok{ }\KeywordTok{c}\NormalTok{(}\FloatTok{2962.9}\NormalTok{,}\FloatTok{2881.2}\NormalTok{,}\FloatTok{2964.1}\NormalTok{,}\FloatTok{2970.0}\NormalTok{,}\FloatTok{2986.1}\NormalTok{,}\FloatTok{2869.3}\NormalTok{,}\FloatTok{2682.4}\NormalTok{)}

\NormalTok{pred22 <-}\StringTok{ }\NormalTok{model22$mean$mu}
\NormalTok{res22 <-}\StringTok{ }\NormalTok{pred22 -}\StringTok{ }\NormalTok{salary}
\NormalTok{s22 <-}\StringTok{ }\KeywordTok{sum}\NormalTok{(res22^}\DecValTok{2}\NormalTok{,}\DataTypeTok{na.rm=}\OtherTok{TRUE}\NormalTok{)}

\NormalTok{pred21 <-}\StringTok{ }\NormalTok{model21$mean$mu}
\NormalTok{res21 <-}\StringTok{ }\NormalTok{pred21 -}\StringTok{ }\NormalTok{salary}
\NormalTok{s21 <-}\StringTok{ }\KeywordTok{sum}\NormalTok{(res21^}\DecValTok{2}\NormalTok{,}\DataTypeTok{na.rm=}\OtherTok{TRUE}\NormalTok{)}

\NormalTok{pred23 <-}\StringTok{ }\NormalTok{model23$mean$mu}
\NormalTok{res23 <-}\StringTok{ }\NormalTok{pred23 -}\StringTok{ }\NormalTok{salary}
\NormalTok{s23 <-}\StringTok{ }\KeywordTok{sum}\NormalTok{(res23^}\DecValTok{2}\NormalTok{,}\DataTypeTok{na.rm=}\OtherTok{TRUE}\NormalTok{)}

\NormalTok{pred24 <-}\StringTok{ }\NormalTok{model24$mean$mu}
\NormalTok{res24 <-}\StringTok{ }\NormalTok{pred24 -}\StringTok{ }\NormalTok{salary}
\NormalTok{s24 <-}\StringTok{ }\KeywordTok{sum}\NormalTok{(res24^}\DecValTok{2}\NormalTok{,}\DataTypeTok{na.rm=}\OtherTok{TRUE}\NormalTok{)}

\NormalTok{pred241 <-}\StringTok{ }\NormalTok{model241$mean$mu}
\NormalTok{res241 <-}\StringTok{ }\NormalTok{pred241 -}\StringTok{ }\NormalTok{salary}
\NormalTok{s241 <-}\StringTok{ }\KeywordTok{sum}\NormalTok{(res241^}\DecValTok{2}\NormalTok{,}\DataTypeTok{na.rm=}\OtherTok{TRUE}\NormalTok{)}

\NormalTok{pred221 <-}\StringTok{ }\NormalTok{model221$mean$mu}
\NormalTok{res221 <-}\StringTok{ }\NormalTok{pred221 -}\StringTok{ }\NormalTok{salary}
\NormalTok{s221 <-}\StringTok{ }\KeywordTok{sum}\NormalTok{(res221^}\DecValTok{2}\NormalTok{,}\DataTypeTok{na.rm=}\OtherTok{TRUE}\NormalTok{)}

\NormalTok{res2 <-}\StringTok{ }\KeywordTok{c}\NormalTok{(s13,s22,s21,s23,s24,s241,s221)}

\NormalTok{tab2 <-}\StringTok{ }\KeywordTok{data.frame}\NormalTok{(DIC2,res2)}

\KeywordTok{print}\NormalTok{(}\KeywordTok{xtable}\NormalTok{(tab2,}\DataTypeTok{align=}\KeywordTok{c}\NormalTok{(}\StringTok{"c"}\NormalTok{,}\StringTok{"c"}\NormalTok{,}\StringTok{"c"}\NormalTok{),}\DataTypeTok{caption=}\StringTok{"Table for the selection of the hierarchical}
\StringTok{Bayesian models. }\CharTok{\textbackslash{}\textbackslash{}}\StringTok{label\{tableselectionmodel\}"}\NormalTok{))}
\end{Highlighting}
\end{Shaded}

\end{document}
